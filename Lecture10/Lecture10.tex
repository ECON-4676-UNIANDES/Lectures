\documentclass[
  shownotes,
  xcolor={svgnames},
  hyperref={colorlinks,citecolor=DarkBlue,linkcolor=DarkRed,urlcolor=DarkBlue}
  ]{beamer}
\usepackage{animate}
\usepackage{amsmath}
\usepackage{amsfonts}
\usepackage{amssymb}
\usepackage{pifont}
\usepackage{mathpazo}
%\usepackage{xcolor}
\usepackage{multimedia}
\usepackage{fancybox}
\usepackage[para]{threeparttable}
\usepackage{multirow}
\setcounter{MaxMatrixCols}{30}
\usepackage{subcaption}
\usepackage{graphicx}
\usepackage{lscape}
\usepackage[compatibility=false,font=small]{caption}
\usepackage{booktabs}
\usepackage{ragged2e}
\usepackage{chronosys}
\usepackage{appendixnumberbeamer}
\usepackage{animate}
\setbeamertemplate{caption}[numbered]
\usepackage{color}
%\usepackage{times}
\usepackage{tikz}
\usepackage{comment} %to comment
%% BibTeX settings
\usepackage{natbib}
\bibliographystyle{apalike}
\bibpunct{(}{)}{,}{a}{,}{,}
\setbeamertemplate{bibliography item}{[\theenumiv]}

% Defines columns for bespoke tables
\usepackage{array}
\newcolumntype{L}[1]{>{\raggedright\let\newline\\\arraybackslash\hspace{0pt}}m{#1}}
\newcolumntype{C}[1]{>{\centering\let\newline\\\arraybackslash\hspace{0pt}}m{#1}}
\newcolumntype{R}[1]{>{\raggedleft\let\newline\\\arraybackslash\hspace{0pt}}m{#1}}


\usepackage{xfrac}


\usepackage{multicol}
\setlength{\columnsep}{0.5cm}

% Theme and colors
\usetheme{Boadilla}

% I use steel blue and a custom color palette. This defines it.
\definecolor{andesred}{HTML}{af2433}

% Other options
\providecommand{\U}[1]{\protect\rule{.1in}{.1in}}
\usefonttheme{serif}
\setbeamertemplate{itemize items}[default]
\setbeamertemplate{enumerate items}[square]
\setbeamertemplate{section in toc}[circle]

\makeatletter

\definecolor{mybackground}{HTML}{82CAFA}
\definecolor{myforeground}{HTML}{0000A0}

\setbeamercolor{normal text}{fg=black,bg=white}
\setbeamercolor{alerted text}{fg=red}
\setbeamercolor{example text}{fg=black}

\setbeamercolor{background canvas}{fg=myforeground, bg=white}
\setbeamercolor{background}{fg=myforeground, bg=mybackground}

\setbeamercolor{palette primary}{fg=black, bg=gray!30!white}
\setbeamercolor{palette secondary}{fg=black, bg=gray!20!white}
\setbeamercolor{palette tertiary}{fg=white, bg=andesred}

\setbeamercolor{frametitle}{fg=andesred}
\setbeamercolor{title}{fg=andesred}
\setbeamercolor{block title}{fg=andesred}
\setbeamercolor{itemize item}{fg=andesred}
\setbeamercolor{itemize subitem}{fg=andesred}
\setbeamercolor{itemize subsubitem}{fg=andesred}
\setbeamercolor{enumerate item}{fg=andesred}
\setbeamercolor{item projected}{bg=gray!30!white,fg=andesred}
\setbeamercolor{enumerate subitem}{fg=andesred}
\setbeamercolor{section number projected}{bg=gray!30!white,fg=andesred}
\setbeamercolor{section in toc}{fg=andesred}
\setbeamercolor{caption name}{fg=andesred}
\setbeamercolor{button}{bg=gray!30!white,fg=andesred}


\usepackage{fancyvrb}
\newcommand{\VerbBar}{|}
\newcommand{\VERB}{\Verb[commandchars=\\\{\}]}
\DefineVerbatimEnvironment{Highlighting}{Verbatim}{commandchars=\\\{\}}
% Add ',fontsize=\small' for more characters per line
\usepackage{framed}
\definecolor{shadecolor}{RGB}{248,248,248}
\newenvironment{Shaded}{\begin{snugshade}}{\end{snugshade}}
\newcommand{\AlertTok}[1]{\textcolor[rgb]{0.94,0.16,0.16}{#1}}
\newcommand{\AnnotationTok}[1]{\textcolor[rgb]{0.56,0.35,0.01}{\textbf{\textit{#1}}}}
\newcommand{\AttributeTok}[1]{\textcolor[rgb]{0.77,0.63,0.00}{#1}}
\newcommand{\BaseNTok}[1]{\textcolor[rgb]{0.00,0.00,0.81}{#1}}
\newcommand{\BuiltInTok}[1]{#1}
\newcommand{\CharTok}[1]{\textcolor[rgb]{0.31,0.60,0.02}{#1}}
\newcommand{\CommentTok}[1]{\textcolor[rgb]{0.56,0.35,0.01}{\textit{#1}}}
\newcommand{\CommentVarTok}[1]{\textcolor[rgb]{0.56,0.35,0.01}{\textbf{\textit{#1}}}}
\newcommand{\ConstantTok}[1]{\textcolor[rgb]{0.00,0.00,0.00}{#1}}
\newcommand{\ControlFlowTok}[1]{\textcolor[rgb]{0.13,0.29,0.53}{\textbf{#1}}}
\newcommand{\DataTypeTok}[1]{\textcolor[rgb]{0.13,0.29,0.53}{#1}}
\newcommand{\DecValTok}[1]{\textcolor[rgb]{0.00,0.00,0.81}{#1}}
\newcommand{\DocumentationTok}[1]{\textcolor[rgb]{0.56,0.35,0.01}{\textbf{\textit{#1}}}}
\newcommand{\ErrorTok}[1]{\textcolor[rgb]{0.64,0.00,0.00}{\textbf{#1}}}
\newcommand{\ExtensionTok}[1]{#1}
\newcommand{\FloatTok}[1]{\textcolor[rgb]{0.00,0.00,0.81}{#1}}
\newcommand{\FunctionTok}[1]{\textcolor[rgb]{0.00,0.00,0.00}{#1}}
\newcommand{\ImportTok}[1]{#1}
\newcommand{\InformationTok}[1]{\textcolor[rgb]{0.56,0.35,0.01}{\textbf{\textit{#1}}}}
\newcommand{\KeywordTok}[1]{\textcolor[rgb]{0.13,0.29,0.53}{\textbf{#1}}}
\newcommand{\NormalTok}[1]{#1}
\newcommand{\OperatorTok}[1]{\textcolor[rgb]{0.81,0.36,0.00}{\textbf{#1}}}
\newcommand{\OtherTok}[1]{\textcolor[rgb]{0.56,0.35,0.01}{#1}}
\newcommand{\PreprocessorTok}[1]{\textcolor[rgb]{0.56,0.35,0.01}{\textit{#1}}}
\newcommand{\RegionMarkerTok}[1]{#1}
\newcommand{\SpecialCharTok}[1]{\textcolor[rgb]{0.00,0.00,0.00}{#1}}
\newcommand{\SpecialStringTok}[1]{\textcolor[rgb]{0.31,0.60,0.02}{#1}}
\newcommand{\StringTok}[1]{\textcolor[rgb]{0.31,0.60,0.02}{#1}}
\newcommand{\VariableTok}[1]{\textcolor[rgb]{0.00,0.00,0.00}{#1}}
\newcommand{\VerbatimStringTok}[1]{\textcolor[rgb]{0.31,0.60,0.02}{#1}}
\newcommand{\WarningTok}[1]{\textcolor[rgb]{0.56,0.35,0.01}{\textbf{\textit{#1}}}}
\usepackage{graphicx}
\makeatletter

\usepackage{tikz}
% Tikz settings optimized for causal graphs.
\usetikzlibrary{shapes,decorations,arrows,calc,arrows.meta,fit,positioning}
\tikzset{
    -Latex,auto,node distance =1 cm and 1 cm,semithick,
    state/.style ={ellipse, draw, minimum width = 0.7 cm},
    point/.style = {circle, draw, inner sep=0.04cm,fill,node contents={}},
    bidirected/.style={Latex-Latex,dashed},
    el/.style = {inner sep=2pt, align=left, sloped}
}


\makeatother






%%%%%%%%%%%%%%% BEGINS DOCUMENT %%%%%%%%%%%%%%%%%%

\begin{document}

\title[Lecture 10]{Lecture 10: \\ Intro to Spatial Data}
\subtitle{Big Data and Machine Learning for Applied Economics \\ Econ 4676}
\date{\today}

\author[Sarmiento-Barbieri]{Ignacio Sarmiento-Barbieri}
\institute[Uniandes]{Universidad de los Andes}


\begin{frame}[noframenumbering]
\maketitle
\end{frame}

%%%%%%%%%%%%%%%%%%%%%%%%%%%%%%%%%%%


%----------------------------------------------------------------------%
\begin{frame}
\frametitle{Announcement }


\begin{itemize} 
    \item {\bf Problem Set 1 is due next Tuesday September 15 at 11:00} 
    \bigskip
    \item At some point over the weekend I'll send what points everyone should present
    \bigskip
    \item Assignment would be based on the groups created on \texttt{Github}
    \bigskip
    \item  You should consider class presentations as mini-seminars, just 2-5 minutes using one or two transparencies
    \bigskip
    \item  Attempt to make a concise interpretation of the relevant material, making effective use of supporting numerical and graphical evidence.    
\end{itemize}
\end{frame}

%----------------------------------------------------------------------% 

\begin{frame}
\frametitle{Agenda}

\tableofcontents

\end{frame}




%----------------------------------------------------------------------%
\section{Motivation }
%----------------------------------------------------------------------%


\begin{frame}[fragile]
\frametitle{Motivation}


\begin{itemize}
  \item In Big Data  volume was only a part of the story 
  \bigskip
  \item Big Data are data of high complexity: anarchic and spontaneous
  \bigskip
  \item They are the by product of an action: pay with credit card, tweet, move from point A to point B, buy a house, etc.
	\bigskip
  \item Now we are going to center on spatial data
\end{itemize}

\end{frame}


%----------------------------------------------------------------------%
\begin{frame}[fragile]
\frametitle{Motivation}

\begin{figure}[H] \centering
  \centering
  \includegraphics[scale=0.3]{figures/blumenstock_fig2.png}
  \\
  \tiny Blumenstock et al (2015)
\end{figure}

\end{frame}


%----------------------------------------------------------------------%
\begin{frame}[fragile]
\frametitle{Motivation}

\begin{figure}[H] \centering
  \centering
  \includegraphics[scale=0.5]{figures/lee_rwanda}
  \\
  \tiny Lee, K., \& Braithwaite, J. (2020)
\end{figure}


\end{frame}

%----------------------------------------------------------------------%
\section{Types of Spatial Data}
%----------------------------------------------------------------------%
\begin{frame}[fragile]
\frametitle{Types of Spatial Data}
Spatial data comes in many ``shapes'' and ``sizes'', the most common types of spatial data are:
\medskip
\begin{itemize}
	\item Points are the most basic form of spatial data. Denotes a single point location, such as cities, a GPS reading or any other discrete object defined in space.
	\medskip
	\item Lines are a set of ordered points, connected by straight line segments
	\medskip
	\item Polygons denote an area, and can be thought as a sequence of connected points, where the first point is the same as the last
	\medskip
	\item Grid (Raster) are a collection of points or rectangular cells, organized in a regular lattice
\end{itemize}

\end{frame}
%----------------------------------------------------------------------%
\begin{frame}[fragile]
\frametitle{Types of Spatial Data: Points}

\begin{figure}[H] \centering
  
\includegraphics[scale=0.35]{figures/albouy_et_al_fig1.png}
  \\
  \tiny Albouy, Christensen \& Sarmiento-Barbieri (2020)
\end{figure}


\end{frame}

%----------------------------------------------------------------------%
\begin{frame}[fragile]
\frametitle{Types of Spatial Data: Lines}

\begin{figure}[H] 
  \centering
\includegraphics[scale=0.35]{figures/safe_passage.png}
  \\
\tiny McMillen, Sarmiento-Barbieri \& Singh, 2019

\end{figure}

\end{frame}

%----------------------------------------------------------------------%
\begin{frame}[fragile]
\frametitle{Types of Spatial Data: Polygons}

\begin{figure}[H] \centering
  \centering
\includegraphics[scale=0.60]{figures/upz.pdf}
  \\
  \tiny Source:\url{https://datosabiertos.bogota.gov.co/dataset/unidad-de-planeamiento-bogota-d-c}
\end{figure}



\end{frame}

%----------------------------------------------------------------------%
\begin{frame}[fragile]
\frametitle{Types of Spatial Data: Combination}

\begin{figure}[H] \centering
  \centering
  
\begin{subfigure}{0.45\linewidth}
\includegraphics[scale=0.08]{figures/ZIP_A.png}
\end{subfigure}
\begin{subfigure}{0.45\linewidth}
\includegraphics[scale=0.08]{figures/ZIP_B.png}

\end{subfigure}
  \\
  \tiny Christensen,Sarmiento-Barbieri  \& Timmins (2020)
\end{figure}


\end{frame}
%----------------------------------------------------------------------%
\begin{frame}[fragile]
\frametitle{Types of Spatial Data: Rasters}


\begin{figure}[H] \centering
  \centering
\includegraphics[scale=0.40]{figures/NewYork.pdf}
  \\
  \tiny Source:\url{https://data.cityofnewyork.us/Environment/NYCCAS-Air-Pollution-Rasters/q68s-8qxv}
\end{figure}



\end{frame}


%----------------------------------------------------------------------%
\section{Reading and Mapping spatial data in R}
%----------------------------------------------------------------------%
\begin{frame}[fragile]
\frametitle{Reading and Mapping spatial data in R}

\begin{itemize}
 \item Spatial data in various formats. 
 \bigskip
 \item One of the most used format are \texttt{shapefiles}
 \bigskip
 \item This type of files stores non topological geometry and attribute information for the spatial features in a data set
 \bigskip
\begin{itemize}

\item Main file: file.shp
\item Index file: file.shx
\item dBASE table: file.dbf
\end{itemize}
\bigskip
\item Today I'm going to use data from \url{https://datosabiertos.bogota.gov.co}
\end{itemize}
\end{frame}
%----------------------------------------------------------------------%
\begin{frame}[fragile]
\frametitle{Reading shapefiles in R}

\begin{itemize}
	\item Basic Packages
	\begin{itemize}
		\item Read and handle spatial data
		\begin{scriptsize}
		\begin{Shaded}
		\begin{Highlighting}[]
\KeywordTok{require}\NormalTok{(}\StringTok{"sf"}\NormalTok{)}
		\end{Highlighting}
		\end{Shaded}
		\end{scriptsize}
		\item Plotting and data wrangling
		\begin{scriptsize}
		\begin{Shaded}
		\begin{Highlighting}[]
\KeywordTok{require}\NormalTok{(}\StringTok{"ggplot2"}\NormalTok{)}
\KeywordTok{require}\NormalTok{(}\StringTok{"dplyr"}\NormalTok{)}
		\end{Highlighting}
		\end{Shaded}
		\end{scriptsize}
	\end{itemize}	
\end{itemize}

\begin{scriptsize}
\begin{Shaded}
\begin{Highlighting}[]
\NormalTok{bars\textless{}{-}}\KeywordTok{st\_read}\NormalTok{(}\StringTok{"egba/EGBa.shp"}\NormalTok{)}
\end{Highlighting}
\end{Shaded}

\begin{verbatim}
## Reading layer `EGBa' from data source `egba/EGBa.shp' using driver 
## `ESRI Shapefile'
## Simple feature collection with 515 features and 7 fields
## geometry type:  POINT
## dimension:      XY
## bbox:           xmin: -74.17607 ymin: 4.577897 xmax: -74.02929 ymax: 4.806253
## CRS:            4686
\end{verbatim}

\end{scriptsize}

\end{frame}
%----------------------------------------------------------------------%
\begin{frame}[fragile]
\frametitle{Visualizing Points}

\begin{minipage}[t]{0.52\linewidth}
        \begin{scriptsize}
           \begin{Shaded}
			\begin{Highlighting}[]
\KeywordTok{ggplot}\NormalTok{()}\OperatorTok{+}
\StringTok{  }\KeywordTok{geom\_sf}\NormalTok{(}\DataTypeTok{data=}\NormalTok{bars) }\OperatorTok{+}
\StringTok{  }\KeywordTok{theme\_bw}\NormalTok{() }\OperatorTok{+}
\StringTok{  }\KeywordTok{theme}\NormalTok{(}\DataTypeTok{axis.title =}\KeywordTok{element\_blank}\NormalTok{(),}
\DataTypeTok{panel.grid.major =} \KeywordTok{element\_blank}\NormalTok{(),}
\DataTypeTok{panel.grid.minor =} \KeywordTok{element\_blank}\NormalTok{(),}
\DataTypeTok{axis.text =} \KeywordTok{element\_text}\NormalTok{(}\DataTypeTok{size=}\DecValTok{6}\NormalTok{))}
			\end{Highlighting}
			\end{Shaded}
			            
        \end{scriptsize}
    \end{minipage}
    \hfill
    \begin{minipage}[t]{0.43\linewidth}%
        \begin{figure}[H] \centering
            \captionsetup{justification=centering}  
            \includegraphics[scale=0.6]{figures/unnamed-chunk-1-1.pdf}
    \end{figure}
    \end{minipage}



\end{frame}
%----------------------------------------------------------------------%
\begin{frame}[fragile]
\frametitle{Visualizing Lines}



\begin{minipage}[t]{0.52\linewidth}
        \begin{scriptsize}

\begin{Shaded}
\begin{Highlighting}[]
\NormalTok{ciclovias\textless{}{-}}\KeywordTok{read\_sf}\NormalTok{(}\StringTok{"Ciclovia/Ciclovia.shp"}\NormalTok{)}
\KeywordTok{ggplot}\NormalTok{()}\OperatorTok{+}
\StringTok{  }\KeywordTok{geom\_sf}\NormalTok{(}\DataTypeTok{data=}\NormalTok{ciclovias) }\OperatorTok{+}
\StringTok{  }\KeywordTok{theme\_bw}\NormalTok{() }\OperatorTok{+}
\StringTok{  }\KeywordTok{theme}\NormalTok{(}\DataTypeTok{axis.title =}\KeywordTok{element\_blank}\NormalTok{(),}
        \DataTypeTok{panel.grid.major =} \KeywordTok{element\_blank}\NormalTok{(),}
        \DataTypeTok{panel.grid.minor =} \KeywordTok{element\_blank}\NormalTok{(),}
        \DataTypeTok{axis.text =} \KeywordTok{element\_text}\NormalTok{(}\DataTypeTok{size=}\DecValTok{6}\NormalTok{))}
\end{Highlighting}
\end{Shaded}
   \end{scriptsize}
    \end{minipage}
    \hfill
    \begin{minipage}[t]{0.43\linewidth}%
        \begin{figure}[H] \centering
            \captionsetup{justification=centering}  

			\includegraphics[scale=0.6]{figures/unnamed-chunk-2-1.pdf}
 \end{figure}
    \end{minipage}

\end{frame}
%----------------------------------------------------------------------%
\begin{frame}[fragile]
\frametitle{Visualizing  Polygons}


\begin{minipage}[t]{0.52\linewidth}
        \begin{scriptsize}

\begin{Shaded}
\begin{Highlighting}[]
\NormalTok{upla\textless{}{-}}\KeywordTok{read\_sf}\NormalTok{(}\StringTok{"upla/UPla.shp"}\NormalTok{)}

\KeywordTok{ggplot}\NormalTok{()}\OperatorTok{+}
\StringTok{  }\KeywordTok{geom\_sf}\NormalTok{(}\DataTypeTok{data=}\NormalTok{upla, }\KeywordTok{aes}\NormalTok{(}\DataTypeTok{fill =}\NormalTok{ UPlArea)) }\OperatorTok{+}
\StringTok{  }\KeywordTok{theme\_bw}\NormalTok{() }\OperatorTok{+}
\StringTok{  }\KeywordTok{theme}\NormalTok{(}\DataTypeTok{axis.title =}\KeywordTok{element\_blank}\NormalTok{(),}
        \DataTypeTok{panel.grid.major =} \KeywordTok{element\_blank}\NormalTok{(),}
        \DataTypeTok{panel.grid.minor =} \KeywordTok{element\_blank}\NormalTok{(),}
        \DataTypeTok{axis.text =} \KeywordTok{element\_text}\NormalTok{(}\DataTypeTok{size=}\DecValTok{6}\NormalTok{))}
\end{Highlighting}
\end{Shaded}
  \end{scriptsize}
    \end{minipage}
    \hfill
    \begin{minipage}[t]{0.43\linewidth}%
        \begin{figure}[H] \centering
            \captionsetup{justification=centering}

\includegraphics[scale=0.6]{figures/unnamed-chunk-2-2.pdf}
 \end{figure}
    \end{minipage}
\end{frame}
%----------------------------------------------------------------------%
\begin{frame}[fragile]
\frametitle{Visualizing Points, Lines, and Polygons}


\begin{minipage}[t]{0.52\linewidth}
        \begin{scriptsize}
\begin{Shaded}
\begin{Highlighting}[]
\KeywordTok{ggplot}\NormalTok{()}\OperatorTok{+}
\StringTok{  }\KeywordTok{geom\_sf}\NormalTok{(}\DataTypeTok{data=}\NormalTok{upla }
\OperatorTok{\%\textgreater{}\%}\StringTok{ }\KeywordTok{filter}\NormalTok{(}\KeywordTok{grepl}\NormalTok{(}\StringTok{"RIO"}\NormalTok{,UPlNombre)}\OperatorTok{==}\OtherTok{FALSE}\NormalTok{), }
\DataTypeTok{fill =} \OtherTok{NA}\NormalTok{) }\OperatorTok{+}
\StringTok{  }\KeywordTok{geom\_sf}\NormalTok{(}\DataTypeTok{data=}\NormalTok{ciclovias, }\DataTypeTok{col=}\StringTok{"red"}\NormalTok{) }\OperatorTok{+}
\StringTok{  }\KeywordTok{geom\_sf}\NormalTok{(}\DataTypeTok{data=}\NormalTok{bars) }\OperatorTok{+}
\StringTok{  }\KeywordTok{theme\_bw}\NormalTok{() }\OperatorTok{+}
\StringTok{  }\KeywordTok{theme}\NormalTok{(}\DataTypeTok{axis.title =}\KeywordTok{element\_blank}\NormalTok{(),}
        \DataTypeTok{panel.grid.major =} \KeywordTok{element\_blank}\NormalTok{(),}
        \DataTypeTok{panel.grid.minor =} \KeywordTok{element\_blank}\NormalTok{(),}
        \DataTypeTok{axis.text =} \KeywordTok{element\_text}\NormalTok{(}\DataTypeTok{size=}\DecValTok{6}\NormalTok{))}
\end{Highlighting}
\end{Shaded}
  \end{scriptsize}
    \end{minipage}
    \hfill
    \begin{minipage}[t]{0.43\linewidth}%
       \medskip
        \begin{figure}[H] \centering
            \captionsetup{justification=centering}
\includegraphics[scale=0.6]{figures/unnamed-chunk-3-1.pdf}

 \end{figure}
    \end{minipage}
\end{frame}

%----------------------------------------------------------------------%
\section{Projections}
%----------------------------------------------------------------------%
\begin{frame}[fragile]
\frametitle{The earth ain't flat}
\begin{itemize}
	\footnotesize
	\item The world is an irregularly shaped ellipsoid, but plotting devices are flat
	\medskip
	\item But if you want to show it on a flat map you need a map projection, 
	\medskip
	\item This  will determine how to transform and distort latitudes and longitudes to preserve some of the map properties: area, shape, distance, direction or bearing
\end{itemize}

\begin{figure}[H] \centering
            \captionsetup{justification=centering}
				\includegraphics[scale=0.6]{figures/world-1.png}
 \end{figure}

\end{frame}
%----------------------------------------------------------------------%
\begin{frame}[fragile]
\frametitle{The earth ain't flat}
\begin{itemize}
	\footnotesize
	\item For example, sailors use Mercator projection where meridians and parallels cross each other always at the same 90 degrees angle.
	\medskip
	\item It allows to easy locate your self on the line showing direction in which you sail
	\medskip
	\item But the projection do not preserve  distances
\end{itemize}
 

\begin{figure}[H] \centering
            \captionsetup{justification=centering}
				\includegraphics[scale=0.3]{figures/Mercator}
				\\
				\tiny
				Source: \url{https://www.geoawesomeness.com/all-map-projections-in-compared-and-visualized/}
 \end{figure}


\end{frame}
%----------------------------------------------------------------------%
\begin{frame}[fragile]
\frametitle{Which projection should I choose?}

\begin{itemize}
\small
\item “There exist no all-purpose projections, all involve distortion when far from the center of the specified frame” (Bivand, Pebesma, and Gómez-Rubio 2013)
\medskip
   \item  Geographic coordinate systems: coordinate systems that span the entire globe (e.g. latitude / longitude).
   \begin{itemize}
   	\footnotesize
    \item For geographic CRSs, the answer is often WGS84
    \item WGS84 is the most common CRS in the world,  EPSG code: 4326. understood.
   \end{itemize}
   \medskip
   \item  Projected coordinate systems: coordinate systems that are localized to minimize visual distortion in a particular region (e.g. Robinson, UTM, State Plane)
\begin{itemize}
	\footnotesize
    \item In some cases, it is not something that we are free to decide: “often the choice of projection is made by a public mapping agency” (Bivand, Pebesma, and Gómez-Rubio 2013).
    \item  This means that when working with local data sources, it is likely preferable to work with the CRS in which the data was provided.
    \item For Bogotá the IGAC promotes the adoption of MAGNA-SIRGAS. EPSG code: 4626
    \end{itemize}




\end{itemize}
\end{frame}
%----------------------------------------------------------------------%
\section{Creating Spatial Objects}
%----------------------------------------------------------------------%
\begin{frame}[fragile]
\frametitle{Creating Spatial Objects}

\begin{scriptsize}
\begin{Shaded}
\begin{Highlighting}[]
\NormalTok{db\textless{}{-}}\KeywordTok{data.frame}\NormalTok{(}\DataTypeTok{place=}\KeywordTok{c}\NormalTok{(}\StringTok{"Uniandes"}\NormalTok{,}\StringTok{"Banco de La Republica"}\NormalTok{),}
		\DataTypeTok{lat=}\KeywordTok{c}\NormalTok{(}\FloatTok{4.601590}\NormalTok{,}\FloatTok{4.602151}\NormalTok{), }
		\DataTypeTok{long=}\KeywordTok{c}\NormalTok{(}\OperatorTok{{-}}\FloatTok{74.066391}\NormalTok{,}\OperatorTok{{-}}\FloatTok{74.072350}\NormalTok{), }
		\DataTypeTok{nudge\_y=}\KeywordTok{c}\NormalTok{(}\OperatorTok{{-}}\FloatTok{0.001}\NormalTok{,}\FloatTok{0.001}\NormalTok{))}
\NormalTok{db\textless{}{-}db }\OperatorTok{\%\textgreater{}\%}\StringTok{ }\KeywordTok{mutate}\NormalTok{(}\DataTypeTok{latp=}\NormalTok{lat,}\DataTypeTok{longp=}\NormalTok{long)}
\NormalTok{db\textless{}{-}}\KeywordTok{st\_as\_sf}\NormalTok{(db,}\DataTypeTok{coords=}\KeywordTok{c}\NormalTok{(}\StringTok{\textquotesingle{}longp\textquotesingle{}}\NormalTok{,}\StringTok{\textquotesingle{}latp\textquotesingle{}}\NormalTok{),}\DataTypeTok{crs=}\DecValTok{4326}\NormalTok{)}
\end{Highlighting}
\end{Shaded}
\end{scriptsize}
\begin{minipage}[t]{0.52\linewidth}
        \begin{tiny}
        \begin{Shaded}
\begin{Highlighting}[]
\KeywordTok{ggplot}\NormalTok{()}\OperatorTok{+}
\StringTok{  }\KeywordTok{geom\_sf}\NormalTok{(}\DataTypeTok{data=}\NormalTok{upla }
	\OperatorTok{\%\textgreater{}\%}\StringTok{ }\KeywordTok{filter}\NormalTok{(UPlNombre}
	\OperatorTok{\%in\%}\KeywordTok{c}\NormalTok{(}\StringTok{"LA CANDELARIA"}\NormalTok{,}\StringTok{"LAS NIEVES"}\NormalTok{)), }\DataTypeTok{fill =} \OtherTok{NA}\NormalTok{) }\OperatorTok{+}
\StringTok{  }\KeywordTok{geom\_sf}\NormalTok{(}\DataTypeTok{data=}\NormalTok{db, }\DataTypeTok{col=}\StringTok{"red"}\NormalTok{) }\OperatorTok{+}
\StringTok{  }\KeywordTok{geom\_label}\NormalTok{(}\DataTypeTok{data =}\NormalTok{ db, }\KeywordTok{aes}\NormalTok{(}\DataTypeTok{x =}\NormalTok{ long, }\DataTypeTok{y =}\NormalTok{ lat, }
				\DataTypeTok{label =}\NormalTok{ place), }
        \DataTypeTok{size =} \DecValTok{3}\NormalTok{, }\DataTypeTok{col =} \StringTok{"black"}\NormalTok{, }\DataTypeTok{fontface =} \StringTok{"bold"}\NormalTok{, }
        \DataTypeTok{nudge\_y =}\NormalTok{db}\OperatorTok{$}\NormalTok{nudge\_y) }\OperatorTok{+}
\StringTok{  }\KeywordTok{theme\_bw}\NormalTok{() }\OperatorTok{+}
\StringTok{  }\KeywordTok{theme}\NormalTok{(}\DataTypeTok{axis.title =}\KeywordTok{element\_blank}\NormalTok{(),}
        \DataTypeTok{panel.grid.major =} \KeywordTok{element\_blank}\NormalTok{(),}
        \DataTypeTok{panel.grid.minor =} \KeywordTok{element\_blank}\NormalTok{(),}
        \DataTypeTok{axis.text =} \KeywordTok{element\_text}\NormalTok{(}\DataTypeTok{size=}\DecValTok{6}\NormalTok{))}
\end{Highlighting}
\end{Shaded}
  \end{tiny}
    \end{minipage}
    \hfill
    \begin{minipage}[t]{0.43\linewidth}%
       \medskip
        \begin{figure}[H] \centering
            \captionsetup{justification=centering}
\includegraphics[scale=0.43]{figures/unnamed-chunk-4-1.pdf}

	\end{figure}
    \end{minipage}

\end{frame}

%----------------------------------------------------------------------%
\section{Measuring Distances}
%----------------------------------------------------------------------%
\begin{frame}[fragile]
\frametitle{Measuring Distances}

\begin{Shaded}
\begin{Highlighting}[]
\KeywordTok{st\_distance}\NormalTok{(db)}
\end{Highlighting}
\end{Shaded}
\begin{scriptsize}
\begin{verbatim}
## Units: [m]
##          [,1]     [,2]
## [1,]   0.0000 664.1323
## [2,] 664.1323   0.0000
\end{verbatim}
\end{scriptsize}

\begin{figure}[H] \centering
  \centering
\includegraphics[scale=0.25]{figures/distance_google_maps}
  \\
  \end{figure}



\end{frame}
%----------------------------------------------------------------------%
\begin{frame}[fragile]
\frametitle{Measuring Distances}

\begin{scriptsize}
\begin{Shaded}
\begin{Highlighting}[]
\KeywordTok{st\_distance}\NormalTok{(db,ciclovias)}
\ErrorTok{Error in st_distance(db, ciclovias) : st_crs(x) == st_crs(y) is not TRUE}
\KeywordTok{st\_crs}\NormalTok{(ciclovias)}
\end{Highlighting}
\end{Shaded}
\end{scriptsize}
\begin{tiny}
\begin{verbatim}
## Coordinate Reference System:
##   User input: 3857 
##   wkt:
## PROJCS["WGS 84 / Pseudo-Mercator",
##     GEOGCS["WGS 84",
##         DATUM["WGS_1984",
##             SPHEROID["WGS 84",6378137,298.257223563,
##                 AUTHORITY["EPSG","7030"]],
##             AUTHORITY["EPSG","6326"]],
##         PRIMEM["Greenwich",0,
##             AUTHORITY["EPSG","8901"]],
##         UNIT["degree",0.0174532925199433,
##             AUTHORITY["EPSG","9122"]],
##         AUTHORITY["EPSG","4326"]],
##     PROJECTION["Mercator_1SP"],
##     PARAMETER["central_meridian",0],
##     PARAMETER["scale_factor",1],
##     PARAMETER["false_easting",0],
##     PARAMETER["false_northing",0],
##     UNIT["metre",1,
##         AUTHORITY["EPSG","9001"]],
##     AXIS["X",EAST],
##     AXIS["Y",NORTH],
##     EXTENSION["PROJ4","+proj=merc +a=6378137 +b=6378137 +lat_ts=0.0 +lon_0=0.0 +x_0=0.0 +y_0=0 +k=1.0 +units=m +nadgrids=@null +wktext +no_defs"],
##     AUTHORITY["EPSG","3857"]]
\end{verbatim}
\end{tiny}
\end{frame}
%----------------------------------------------------------------------%
\begin{frame}[fragile]
\frametitle{Measuring Distances}

\begin{scriptsize}
\begin{Shaded}
\begin{Highlighting}[]
\NormalTok{ciclovias\textless{}{-}}\KeywordTok{st\_transform}\NormalTok{(ciclovias, }\DecValTok{4686}\NormalTok{)}
\KeywordTok{st\_crs}\NormalTok{(ciclovias)}
\end{Highlighting}
\end{Shaded}
\begin{tiny}
\begin{verbatim}
## Coordinate Reference System:
##   User input: EPSG:4686 
##   wkt:
## GEOGCS["MAGNA-SIRGAS",
##     DATUM["Marco_Geocentrico_Nacional_de_Referencia",
##         SPHEROID["GRS 1980",6378137,298.257222101,
##             AUTHORITY["EPSG","7019"]],
##         TOWGS84[0,0,0,0,0,0,0],
##         AUTHORITY["EPSG","6686"]],
##     PRIMEM["Greenwich",0,
##         AUTHORITY["EPSG","8901"]],
##     UNIT["degree",0.0174532925199433,
##         AUTHORITY["EPSG","9122"]],
##     AUTHORITY["EPSG","4686"]]
\end{verbatim}
\end{tiny}
\begin{Shaded}
\begin{Highlighting}[]
\NormalTok{db\textless{}{-}}\KeywordTok{st\_transform}\NormalTok{(db, }\DecValTok{4326}\NormalTok{)}
\KeywordTok{st\_distance}\NormalTok{(db,ciclovias)}
\end{Highlighting}
\end{Shaded}

\begin{verbatim}
## Units: [m]
##          [,1]     [,2]     [,3]     [,4]     [,5]     [,6]     [,7]     [,8]
## [1,] 9514.617 10789.90 6035.283 12855.90 6025.017 8311.922 4579.450 741.6047
## [2,] 9221.998 10686.39 6143.960 13004.84 5871.073 7656.183 4014.993 116.5939
##           [,9]    [,10]    [,11]    [,12]    [,13]    [,14]
## [1,] 1002.8751 6255.692 2385.125 8402.580 8669.030 3788.265
## [2,]  981.1991 5839.565 2425.508 7738.774 8048.108 3436.819
\end{verbatim}
\end{scriptsize}
\end{frame}
%----------------------------------------------------------------------%
\begin{frame}[fragile]
\frametitle{Measuring Distances}


\begin{minipage}[t]{0.52\linewidth}
        \begin{scriptsize}
\begin{Shaded}
\begin{Highlighting}[]
\NormalTok{ciclovias\_sp\textless{}{-}ciclovias[}\DecValTok{8}\NormalTok{,]}

\KeywordTok{ggplot}\NormalTok{()}\OperatorTok{+}
\StringTok{  }\KeywordTok{geom\_sf}\NormalTok{(}\DataTypeTok{data=}\NormalTok{ciclovias[}\DecValTok{8}\NormalTok{,], }\DataTypeTok{fill =} \OtherTok{NA}\NormalTok{) }\OperatorTok{+}
\StringTok{  }\KeywordTok{geom\_sf}\NormalTok{(}\DataTypeTok{data=}\NormalTok{db, }\DataTypeTok{col=}\StringTok{"red"}\NormalTok{) }\OperatorTok{+}
\StringTok{  }\KeywordTok{theme\_bw}\NormalTok{() }\OperatorTok{+}
\StringTok{  }\KeywordTok{theme}\NormalTok{(}\DataTypeTok{axis.title =}\KeywordTok{element\_blank}\NormalTok{(),}
        \DataTypeTok{panel.grid.major =} \KeywordTok{element\_blank}\NormalTok{(),}
        \DataTypeTok{panel.grid.minor =} \KeywordTok{element\_blank}\NormalTok{(),}
        \DataTypeTok{axis.text =} \KeywordTok{element\_text}\NormalTok{(}\DataTypeTok{size=}\DecValTok{6}\NormalTok{))}
\end{Highlighting}
\end{Shaded}
  \end{scriptsize}
    \end{minipage}
    \hfill
    \begin{minipage}[t]{0.43\linewidth}%
       \medskip
        \begin{figure}[H] \centering
            \captionsetup{justification=centering}
\includegraphics[scale=0.6]{figures/unnamed-chunk-7-1.pdf}

	\end{figure}
    \end{minipage}

\end{frame}



%----------------------------------------------------------------------%

\begin{frame}
\frametitle{Review \& Next Steps}
  
  \begin{itemize} 
    \item Intro to Shapes
    \medskip
    \item Basics in \texttt{R}
    \medskip
    \item Projections
  	\bigskip  

	\item  Next class: Problem Set Presentations


\bigskip  
\item Questions? Questions about software? 

\end{itemize}
\end{frame}

%----------------------------------------------------------------------%
\section{Further Readings}
%----------------------------------------------------------------------%
\begin{frame}
\frametitle{Further Readings}
\scriptsize
\begin{itemize}

   
  \item Arbia, G. (2014). A primer for spatial econometrics with applications in R. Palgrave Macmillan.
  \medskip
  \item Albouy, D., Christensen, P., \& Sarmiento-Barbieri, I. (2020). Unlocking amenities: Estimating public good complementarity. Journal of Public Economics, 182, 104110.
  \medskip
  \item Bivand, R. S.,  \& Pebesma, E. J. (2020). Spatial Data Science \url{https://keen-swartz-3146c4.netlify.app/} (Chapter 8)
  \medskip
  \item Bivand, R. S., Gómez-Rubio, V., \& Pebesma, E. J. (2008). Applied spatial data analysis with R (Vol. 747248717, pp. 237-268). New York: Springer.
  \medskip
  \item Blumenstock, J., Cadamuro, G., \& On, R. (2015). Predicting poverty and wealth from mobile phone metadata. Science, 350(6264), 1073-1076.
  \medskip
  \item Christensen, P.,  Sarmiento-Barbieri, I., Timmins C. (2020). Housing Discrimination and the Pollution Exposure Gap in the United States. NBER WP No. 26805
  \medskip
  \item Lee, K., \& Braithwaite, J. (2020). High-Resolution Poverty Maps in Sub-Saharan Africa. arXiv preprint arXiv:2009.00544.
  \medskip
  \item Lovelace, R., Nowosad, J., \& Muenchow, J. (2019). Geocomputation with R. CRC Press. (Chapters 2 \& 6)
  \medskip
  \item McMillen, D., Sarmiento-Barbieri, I., \& Singh, R. (2019). Do more eyes on the street reduce Crime? Evidence from Chicago's safe passage program. Journal of urban economics, 110, 1-25.
  \medskip
  \item Wasser, L. GIS With R: Projected vs Geographic Coordinate Reference Systems \url{https://www.earthdatascience.org/courses/earth-analytics/spatial-data-r/geographic-vs-projected-coordinate-reference-systems-UTM/} Last Access September 10, 2020
\end{itemize}

\end{frame}






%----------------------------------------------------------------------%
%----------------------------------------------------------------------%
\end{document}
%----------------------------------------------------------------------%
%----------------------------------------------------------------------%

